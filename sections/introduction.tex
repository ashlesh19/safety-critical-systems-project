\label{sec:introduction}
Predictive maintenance is a technique that uses data analysis tools and techniques to detect anomalies in your operation and possible defects in equipment and processes so you can fix them before they result in failure.
Ideally, predictive maintenance allows the maintenance frequency to be as low as possible to prevent unplanned reactive maintenance, without incurring costs associated with doing too much preventive maintenance.
When predictive maintenance is working effectively as a maintenance strategy, maintenance is only performed on machines when it is required. That is, just before failure is likely to occur. This brings several cost savings:

Minimizing the time, the equipment is being maintained
Minimizing the production hours lost to maintenance
Minimizing the cost of spare parts and supplies
These cost savings come at a price, however. Some condition monitoring techniques are expensive and require specialist and experienced personnel for data analysis to be effective.
